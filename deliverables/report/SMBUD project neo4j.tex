%----------------------------------------------------------------------------------------
%	PACKETS AND CONFIGURATION
%----------------------------------------------------------------------------------------

\documentclass[12pt, a4paper]{article}

\usepackage{times} % Times New Roman font
\usepackage{graphicx} % 'graphics' package interface
\usepackage{geometry} % Edit document margins
\usepackage{hyperref} % Table of contents hyperlinks
\usepackage{tcolorbox} % Colored boxes for code

% HELPER PACKAGES (REMOVE IN FINAL) %
\usepackage{blindtext} % Lorem Ipsum
\usepackage{todonotes} % TODOs as useful reminders
\setlength{\marginparwidth}{2cm} % Otherwise todonotes gets angry at me lol
\setuptodonotes{fancyline, color=green!40, shadow} % TODOs options
% HELPER PACKAGES (REMOVE IN FINAL) %

\graphicspath{ {./res/} } % Path to graphics
\hypersetup{    % ToC Hyperlink setup
    colorlinks,
    citecolor=black,
    filecolor=black,
    linkcolor=black,
    urlcolor=black
}

%----------------------------------------------------------------------------------------
%	DOCUMENT
%----------------------------------------------------------------------------------------

\begin{document}

\newgeometry{top=7cm, bottom=2cm} % Setting the margins for the title

% Title
\begin{titlepage}
    \centering
    {\Huge\bfseries Pandemic Information System Model\par} % Project title
    \vspace{1.5cm}
    {\scshape\large Systems and Methods for Big and Unstructured Data \par} % Course
    \vspace{0.5cm}
    {\scshape\large Prof. Marco Brambilla \par} % Professor
    \vspace{1cm}
    {\scshape\large % Description
        First delivery \par 
        Neo4j Project \par 
    }
    \vspace{0.5cm}
    {\slshape\large November 2021 \par} % Date
    \vspace{1cm}
    {\large\itshape % Authors
        \todo{add personal codes}
        Avci Oguzhan - \texttt{xxxxxxxx}\\
        Gentile Nicole - \texttt{10594355}\\
        Rigamonti Davide - \texttt{10629791}\\
        Singh Raul - \texttt{xxxxxxxx}\\
        Tagliaferri Mattia - \texttt{xxxxxxxx}
    }
    \vfill
    \begin{figure}[b]
        \includegraphics[scale=0.6]{polimi} % Polimi logo
        \centering
    \end{figure}

    \pagenumbering{gobble} % Remove page number

\end{titlepage}

\newgeometry{bottom=3cm} % Reset the margins
\pagenumbering{arabic} % Reset the page number

\clearpage

% INDEX
\tableofcontents

% LIST OF TODOs (REMOVE IN FINAL)
\listoftodos

\clearpage

% INTRODUCTION
\section{Introduction}

\subsection{Problem Specification}

We represented a population of XXX \todo{fill with population size} people in 
the USA. \\
The database receives data coming from tracing applications that use sensor in 
smartphones, wearable objects or other devices to understand whether two people 
had a contact; data includes date and time of the contact. \\  
Some places, like restaurants, theaters and hospitals, collect date and name of 
visiting people. \\
The idea is to use this database so that, if a person gets positive, we can 
understand who are all people who had a contact with him/her. Contacts are of 
different types: in family (since people from the same family are always in 
contact), in a location (if someone is positive, we alert people who were in 
the same location on the same day) or are given by an application using sensors. 
Data are recorded from 02/2020 and can be used for analytical purposes. 

\subsection{Hypoteses}

\emph{Vaccine date}, \emph{date of the last contagion}, \emph{date of the last 
negative test} and \emph{healing date} are optional fields. \\ 
A person is considered either \emph{infected} if they have a contagion date, 
\emph{healed} if they have an healing date and a contagion date or neither if 
they have none yet. \\
We assumed that people can do tests without being infected (or after healing) 
and that people can decide not to get the vaccine but vaccinated people can 
still get infected. \\
We also assumed that people can get covid at most once (realistic if we 
consider perfect antibodies), in this way it's easier to store and retrieve 
data in order to build statistics. \\
Members of a family are assumed to be all the people who live together, 
relatives who see each other very often or roommates. Obviously, all members 
of a family live in the same city and are considered "always in contact". \\
When adding data about people going places, we do not consider distances 
between the city where they live in and the location they go to, so they can 
exist in different places at the same time. \todo{check if accesses from the same person can be generated at the same time} \\   
\todo{is this still true?} 
Regarding the places we have: people who got covid after 18/10 went to places 
from 18/09/2021 to 17/10/2021, since we assume that later they are in 
quarantine; all the others went to visit locations from 18/09/2021 to 
17/11/2021. 
If a person becomes positive, we alert all people who were in the same location
in a 4 hour span of time centered on the time of arrival of the infected
person. \\
For simplicity, for people who are in hospital date of hospitalization and 
contagion coincide; we also assume they leave the hospital on the healing 
date. \todo{is this still true?} \\
In order to populate the database, we imposed that every person visits from 6 
to 10 locations.\todo{we could add more} 

\clearpage

% DATABASE
\section{Database}

\subsection{ER Diagram}

\blindtext

\subsection{Dataset description}

We used three types of nodes: Person, Location and City. 
People are characterized by their name (composed by name and surname), 
birthdate, city, email, social security number, vaccine date, date of the last 
contagion, date of the last negative test and healing date. \\
Locations have a name and a type (restaurant, theater, hospital); 
type is used so that the dataset can be easily expanded with new types of 
location. \\
Possible relationships are: 
\begin{itemize}
    \item \texttt{WENT\char`_TO} to track people who visited a location at a 
        certain time;
	\item \texttt{IN\char`_FAMILY} for family relationships;
	\item \texttt{IS\char`_IN} between locations and cities;
	\item \texttt{LIVES\char`_IN} to link people to the city they live in;
    \item \texttt{HAS\char`_MET} to indicate contacts between people using 
        tracing app or devices;
    \item \texttt{IS\char`_HOSPITALIZED\char`_IN} to indicate people who 
        are/were hospitalized for covid reasons.
\end{itemize}

\subsection{Queries}

Only the most significative queries are shown in this document.

\subsubsection{Average age}

\begin{tcolorbox}[fontupper=\scriptsize]
    \begin{verbatim}
WITH    apoc.date.convert(timestamp(), "ms", "d") AS now
MATCH   (pc:Person)
WHERE   pc.contagion_date IS NOT NULL AND
        NOT EXISTS((pc)-[:IS_HOSPITALIZED_IN]->(:Location {type:"Hospital"}))
WITH    apoc.date.parse(toString(pc.birthdate), "d", "yyyy-MM-dd") AS pc_birth,
        now
MATCH   (ph:Person)
WHERE   EXISTS((ph)-[:IS_HOSPITALIZED_IN]->(:Location {type:"Hospital"}))
WITH    apoc.date.parse(toString(ph.birthdate), "d", "yyyy-MM-dd") AS ph_birth,
        now, pc_birth

RETURN  avg((now-pc_birth)/365) AS avg_age_nothospitalized, 
        avg((now-ph_birth)/365) AS avg_age_hospitalized;
    \end{verbatim}
\end{tcolorbox}

\noindent % Removes indentation after box
Returns the average age for infected (but not hospitalized) and for infected 
and hospitalized people inside the database.

\subsubsection{Cities}

\todo{cities query}

\subsubsection{Devices}

\begin{tcolorbox}[fontupper=\scriptsize]
    \begin{verbatim}
MATCH   (p1:Person)-[met:HAS_MET]->(p2:Person)
WITH    apoc.date.parse(toString(p1.contagion_date), "ms", "yyyy-MM-dd") 
            AS p1_contagion_date,
        apoc.date.parse(toString(p1.healing_date), "ms", "yyyy-MM-dd") 
            AS p1_healing_date,
        apoc.date.parse(toString(p2.contagion_date), "ms", "yyyy-MM-dd") 
            AS p2_contagion_date,
        met
WHERE   p1_contagion_date IS NOT NULL AND
        (p2_contagion_date IS NULL OR p2_contagion_date > p1_contagion_date)
WITH    apoc.date.add(p1_contagion_date, "ms", -14, "d") AS contagion_l,
        met, p1_healing_date
WHERE   met.date.EpochMillis >= contagion_l AND
        (p1_healing_date IS NULL OR met.date.EpochMillis < p1_healing_date)
RETURN  met.device, count(DISTINCT met) AS infected_contacts
    \end{verbatim}
\end{tcolorbox}

\noindent % Removes indentation after box
Returns the number of contacts registered for each type of device.

% \subsubsection{Healed}

% \begin{tcolorbox}[fontupper=\scriptsize]
%     \begin{verbatim}
% MATCH   (p:Person)
% WHERE   p.contagion_date IS NOT NULL AND p.healing_date IS NOT NULL AND
%         p.healing_date >= $date1 AND p.healing_date <= $date2
% RETURN  count(p);
%     \end{verbatim}
% \end{tcolorbox}

% \noindent % Removes indentation after box
% Returns the number of healed people over a given span of time. \\
% Takes the following parameters: 
% \begin{itemize}
%     \item \texttt{\$date1} \emph{(date)}: starting date;
% 	\item \texttt{\$date2} \emph{(date)}: ending date.
% \end{itemize}


% \subsubsection{Hospitalized}

\subsubsection{Infected}

\todo{infected query}

\subsubsection{Locations}

\todo{locations query}

\subsubsection{Notifications}

\begin{tcolorbox}[fontupper=\scriptsize]
    \begin{verbatim}
MATCH   (p:Person {ssn:$ssn})
WHERE   p.contagion_date IS NOT NULL
CALL {
        WITH    p
        CALL    apoc.path.expandConfig(p, {
                        relationshipFilter: "FAMILY>",
                        minLevel: 1,
                        maxLevel: -1,
                        labelFilter: "+Person",
                        uniqueness:"NODE_GLOBAL",
                        bfs:true
        }) YIELD path AS path
        UNWIND tail(nodes(path)) AS x
        WITH DISTINCT x AS contact
        RETURN  contact, "family" AS tracing, null AS time

        UNION

        WITH    p
        MATCH   (p)-[met:HAS_MET]->(pc:Person)
        WITH    apoc.date.parse(toString(p.contagion_date), "ms", "yyyy-MM-dd") 
                    AS contagion_date,
                apoc.date.parse(toString(p.healing_date), "ms", "yyyy-MM-dd") 
                    AS healing_date,
                pc, met
        WITH    apoc.date.add(contagion_date, "ms", -14, "d") AS contagion_l,
                pc, met, healing_date
        WHERE   (met.date.EpochMillis >= contagion_l) AND 
                (met.date.EpochMillis < healing_date OR healing_date IS NULL)
        RETURN  pc AS contact, met.device AS tracing, met.date AS time

        UNION

        WITH    p
        MATCH   (p)-[go1:WENT_TO]->(loc:Location)<-[go2:WENT_TO]-(pl:Person)
        WITH    apoc.date.parse(toString(p.contagion_date), "ms", "yyyy-MM-dd") 
                    AS contagion_date,
                apoc.date.parse(toString(p.healing_date), "ms", "yyyy-MM-dd") 
                    AS healing_date,
                pl, go1, go2, loc
        WITH    apoc.date.add(contagion_date, "ms", -14, "d") AS contagion_l,
                apoc.date.add(go1.date.EpochMillis, "ms", -2, "h") AS go1_l,
                apoc.date.add(go1.date.EpochMillis, "ms", 2, "h") AS go1_u,
                pl, go1, go2, loc, healing_date
        WHERE   go2.date.EpochMillis >= go1_l AND 
                go2.date.EpochMillis <= go1_u AND
                go1.date.EpochMillis >= contagion_l AND 
                (go1.date.EpochMillis < healing_date OR healing_date IS NULL)
        RETURN  pl AS contact, loc.name AS tracing, go1.date AS time
}
RETURN DISTINCT contact, collect({tracing:tracing, time:time}) AS info
    \end{verbatim}
\end{tcolorbox}

\noindent % Removes indentation after box
Returns a list of people that a specific infected person (given their 
\emph{social security number}) may have been in contact with; considering 
family relationships, contacts from device tracking and inferred contacts from
explicit data collection in specific locations. \\
For each contact we offer complete information on the person that came in 
contact with the infected and, when possible, place/device information and 
time of the contact. \\
Takes the following parameters: 
\begin{itemize}
    \item \texttt{\$ssn} \emph{(string)}: social security number of the 
        infected person.
\end{itemize}

\subsubsection{Vaccinated}

\todo{vaccinated query (one of the two)}

\subsection{Commands}

Only the most significative commands are shown in this document.

\subsubsection{Contact}

\begin{tcolorbox}[fontupper=\scriptsize]
    \begin{verbatim}
MATCH   (a:Person {name:$name1}), (b:Person {name:$name2})
CREATE  (a)-[r:HAS_MET {date:$date}, {device:$device}]->(b), 
        (b)-[r:HAS_MET {date:$date}, {device:$device}]->(a);
    \end{verbatim}
\end{tcolorbox}

\noindent % Removes indentation after box
Registration of a contact via device.
Takes the following parameters: 
\begin{itemize}
    \item \texttt{\$ssn1} \emph{(string)}: social security number of the 
        first person;
    \item \texttt{\$ssn2} \emph{(string)}: social security number of the
        second person;
    \item \texttt{\$date} \emph{(datetime)}: date and time of the contact;
    \item \texttt{\$device} \emph{(string)}: type of device that detected a 
        contact.
\end{itemize}

\subsubsection{Negative test}

\todo{negative test command}

\subsubsection{Positive test}

\todo{positive test command}

\clearpage

% APPLICATION
\section{Application}

\subsection{Description}

\blindtext

\subsection{User Guide}

\blindtext

\clearpage

% REFERENCES AND SOURCES
\section{References and sources}

\blindtext

\clearpage

\end{document}